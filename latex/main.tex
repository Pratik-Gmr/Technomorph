\documentclass[12pt]{article}
\usepackage{template}
\usepackage{listings}
\usepackage{xcolor}
\usepackage{ulem}
\usepackage{soul}

% Set Libertinus Serif font if using XeLaTeX/LuaLaTeX
\usepackage{iftex}
\ifXeTeX
    \setmainfont{Libertinus Serif}
\else\ifLuaTeX
    \setmainfont{Libertinus Serif}
\fi

% Cover page information
\coverTitle{Day 1: Introduction to Robots and Robotics}
\coverAuthor{Pratik Ghimire}
\coverRollno{PUR0381BCT051}
\coverTo{Scientiac}
\coverLogo{robo.png}
\coverDate{2025-06-30}

\begin{document}

% Generate cover page
\makecover

% Setup main document formatting
\setupmain

\section{How I Got Here}
\subsection{Literally How}

I read the venue — it said \textbf{Robotics Club}. Since I had been there before (it's just beside our TT board), I reached there just fine and attended the session.  
\textit{Just kidding.}

\subsection{The Actual How}

I had always been interested in this club. Even during the orientation, when all the clubs were being introduced, the only one that actually caught my attention was the \textbf{Robotics Club}.  
Although I’m not very familiar with \textbf{technology}, \textbf{robotics}, \textbf{microcontrollers}, and oh god so many scary terms, the notice about an introductory event, \textbf{Technomorph}, really excited me.

This was my opportunity, so I filled out the form and went through great hardships —  
\textit{(They made me write two 200-word essays!)} - and finally got selected as a candidate.


\section{What I Learned}
\subsection{About robots and robotics}
The most mind-boggling thing I learned-it might break your brain as well —\textit{All robots don't have to be human like}, even \textbf{drones} are robots.
On top of that, I learned about the different types of robots and also watched videos of some cool real-life robots.
There also was \textbf{Laws of Robotics} — but we are cool people — we ignore boring bits.

\subsection{About Time Travel}
Then we went beyond robots, we talked about something even more advanced — \textbf{Time Machine}. Version Control System — \textbf{GIT} was introduced and we saw how we could travel back in time — to older versions of a project — with its power. Since this was my first introduction to such a concept, \textit{I was beyond shook}. Then, I was also name-dropped to Github, Gitlab and Bitbucket.

\subsection{About Documentation}
The session then proceeded to the \textit{forbidden realms}. We talked about the one thing that \textit{Every programmer wishes for but no programmer is ready to make}, the \textbf{Documentation}. That is the reason I am even writing this thing right now to be honest. I also got to learn about \textbf{README.md} in GitHub repository and \textbf{Latex} and \textbf{Typst}.

\section{The Assignment}
There were 4 assignments on the first day:
\begin{enumerate}
    \item \textbf{Install Arduino IDE V2}:
    Which was like 2 commands so pretty simple and easy.
    \item \textbf{Search about Git and GitHub}:
    I was already amazed by these when I heard about them in the session so, looking them up was the first thing I did as soon as I reached home.
    \item \textbf{Look up LaTeX and/or Typst}:
    I clearly did that as well, as I am writing this thing in Latex right now.
    \item \textbf{Create a github repo and push Documentation written either in Latex or Typst}:
    This one is in process but will be done soon enough.
\end{enumerate}

\section{Conclusion}
This was a great start to what seems like an exciting journey.  
Looking forward to building cool robots, breaking a few things (on purpose), and learning a lot along the way.

\end{document}
